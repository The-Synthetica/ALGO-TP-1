\documentclass[10pt,a4paper]{article}

\usepackage[spanish,activeacute,es-tabla]{babel}
\usepackage[utf8]{inputenc}
\usepackage{ifthen}
\usepackage{listings}
\usepackage{dsfont}
\usepackage{subcaption}
\usepackage{amsmath}
\usepackage[strict]{changepage}
\usepackage[top=1cm,bottom=2cm,left=1cm,right=1cm]{geometry}%
\usepackage{color}%
\newcommand{\tocarEspacios}{%
	\addtolength{\leftskip}{3em}%
	\setlength{\parindent}{0em}%
}

% Especificacion de procs

\newcommand{\In}{\textsf{in }}
\newcommand{\Out}{\textsf{out }}
\newcommand{\Inout}{\textsf{inout }}

\newcommand{\encabezadoDeProc}[4]{%
	% Ponemos la palabrita problema en tt
	%  \noindent%
	{\normalfont\bfseries\ttfamily proc}%
	% Ponemos el nombre del problema
	\ %
	{\normalfont\ttfamily #2}%
	\
	% Ponemos los parametros
	(#3)%
	\ifthenelse{\equal{#4}{}}{}{%
		% Por ultimo, va el tipo del resultado
		\ : #4}
}

\newenvironment{proc}[4][res]{%
	
	% El parametro 1 (opcional) es el nombre del resultado
	% El parametro 2 es el nombre del problema
	% El parametro 3 son los parametros
	% El parametro 4 es el tipo del resultado
	% Preambulo del ambiente problema
	% Tenemos que definir los comandos requiere, asegura, modifica y aux
	\newcommand{\requiere}[2][]{%
		{\normalfont\bfseries\ttfamily requiere}%
		\ifthenelse{\equal{##1}{}}{}{\ {\normalfont\ttfamily ##1} :}\ %
		\{\ensuremath{##2}\}%
		{\normalfont\bfseries\,\par}%
	}
	\newcommand{\asegura}[2][]{%
		{\normalfont\bfseries\ttfamily asegura}%
		\ifthenelse{\equal{##1}{}}{}{\ {\normalfont\ttfamily ##1} :}\
		\{\ensuremath{##2}\}%
		{\normalfont\bfseries\,\par}%
	}
	\renewcommand{\aux}[4]{%
		{\normalfont\bfseries\ttfamily aux\ }%
		{\normalfont\ttfamily ##1}%
		\ifthenelse{\equal{##2}{}}{}{\ (##2)}\ : ##3\, = \ensuremath{##4}%
		{\normalfont\bfseries\,;\par}%
	}
	\renewcommand{\pred}[3]{%
		{\normalfont\bfseries\ttfamily pred }%
		{\normalfont\ttfamily ##1}%
		\ifthenelse{\equal{##2}{}}{}{\ (##2) }%
		\{%
		\begin{adjustwidth}{+5em}{}
			\ensuremath{##3}
		\end{adjustwidth}
		\}%
		{\normalfont\bfseries\,\par}%
	}
	
	\newcommand{\res}{#1}
	\vspace{1ex}
	\noindent
	\encabezadoDeProc{#1}{#2}{#3}{#4}
	% Abrimos la llave
	\par%
	\tocarEspacios
}
{
	% Cerramos la llave
	\vspace{1ex}
}

\newcommand{\aux}[4]{%
	{\normalfont\bfseries\ttfamily\noindent aux\ }%
	{\normalfont\ttfamily #1}%
	\ifthenelse{\equal{#2}{}}{}{\ (#2)}\ : #3\, = \ensuremath{#4}%
	{\normalfont\bfseries\,;\par}%
}

\newcommand{\pred}[3]{%
	{\normalfont\bfseries\ttfamily\noindent pred }%
	{\normalfont\ttfamily #1}%
	\ifthenelse{\equal{#2}{}}{}{\ (#2) }%
	\{%
	\begin{adjustwidth}{+2em}{}
		\ensuremath{#3}
	\end{adjustwidth}
	\}%
	{\normalfont\bfseries\,\par}%
}

% Tipos

\newcommand{\nat}{\ensuremath{\mathds{N}}}
\newcommand{\ent}{\ensuremath{\mathds{Z}}}
\newcommand{\float}{\ensuremath{\mathds{R}}}
\newcommand{\bool}{\ensuremath{\mathsf{Bool}}}
\newcommand{\cha}{\ensuremath{\mathsf{Char}}}
\newcommand{\str}{\ensuremath{\mathsf{String}}}

% Logica

\newcommand{\True}{\ensuremath{\mathrm{true}}}
\newcommand{\False}{\ensuremath{\mathrm{false}}}
\newcommand{\Then}{\ensuremath{\rightarrow}}
\newcommand{\Iff}{\ensuremath{\leftrightarrow}}
\newcommand{\implica}{\ensuremath{\longrightarrow}}
\newcommand{\IfThenElse}[3]{\ensuremath{\mathsf{if}\ #1\ \mathsf{then}\ #2\ \mathsf{else}\ #3\ \mathsf{fi}}}
\newcommand{\yLuego}{\land _L}
\newcommand{\oLuego}{\lor _L}
\newcommand{\implicaLuego}{\implica _L}

\newcommand{\cuantificador}[5]{%
	\ensuremath{(#2 #3: #4)\ (%
		\ifthenelse{\equal{#1}{unalinea}}{
			#5
		}{
			$ % exiting math mode
			\begin{adjustwidth}{+2em}{}
				$#5$%
			\end{adjustwidth}%
			$ % entering math mode
		}
		)}
}

\newcommand{\existe}[4][]{%
	\cuantificador{#1}{\exists}{#2}{#3}{#4}
}
\newcommand{\paraTodo}[4][]{%
	\cuantificador{#1}{\forall}{#2}{#3}{#4}
}

%listas

\newcommand{\TLista}[1]{\ensuremath{seq \langle #1\rangle}}
\newcommand{\lvacia}{\ensuremath{[\ ]}}
\newcommand{\lv}{\ensuremath{[\ ]}}
\newcommand{\longitud}[1]{\ensuremath{|#1|}}
\newcommand{\cons}[1]{\ensuremath{\mathsf{addFirst}}(#1)}
\newcommand{\indice}[1]{\ensuremath{\mathsf{indice}}(#1)}
\newcommand{\conc}[1]{\ensuremath{\mathsf{concat}}(#1)}
\newcommand{\cab}[1]{\ensuremath{\mathsf{head}}(#1)}
\newcommand{\cola}[1]{\ensuremath{\mathsf{tail}}(#1)}
\newcommand{\sub}[1]{\ensuremath{\mathsf{subseq}}(#1)}
\newcommand{\en}[1]{\ensuremath{\mathsf{en}}(#1)}
\newcommand{\cuenta}[2]{\mathsf{cuenta}\ensuremath{(#1, #2)}}
\newcommand{\suma}[1]{\mathsf{suma}(#1)}
\newcommand{\twodots}{\ensuremath{\mathrm{..}}}
\newcommand{\masmas}{\ensuremath{++}}
\newcommand{\matriz}[1]{\TLista{\TLista{#1}}}
\newcommand{\seqchar}{\TLista{\cha}}

\renewcommand{\lstlistingname}{Código}
\lstset{% general command to set parameter(s)
	language=Java,
	morekeywords={endif, endwhile, skip},
	basewidth={0.47em,0.40em},
	columns=fixed, fontadjust, resetmargins, xrightmargin=5pt, xleftmargin=15pt,
	flexiblecolumns=false, tabsize=4, breaklines, breakatwhitespace=false, extendedchars=true,
	numbers=left, numberstyle=\tiny, stepnumber=1, numbersep=9pt,
	frame=l, framesep=3pt,
	captionpos=b,
}

\usepackage{caratula} % Version modificada para usar las macros de algo1 de ~> https://github.com/bcardiff/dc-tex


\titulo{Trabajo practico 1: Especificacion y WP}
\subtitulo{Primer cuatrimestre de 2024}

\fecha{\today}

\materia{Algoritmos y Estructuras de Datos - DC - UBA}
\grupo{Grupo: EVLUAFGUEYVCXPKHDNUP}

\integrante{Curti, Nahuel}{97/23}{nahuel0curti@gmail.com}
\integrante{Dosio, Martin}{291/23}{dosiomartin@gmail.com}
\integrante{Lemes, Tiziano}{796/23}{tizilemes@gmail.com}
\integrante{Rizzi, Francisco}{766/23}{rizzifranciscojose@gmail.com}
% Pongan cuantos integrantes quieran

% Declaramos donde van a estar las figuras
% No es obligatorio, pero suele ser comodo
\graphicspath{{../static/}}

\begin{document}

\maketitle

\section{Especificacion}




\subsection*{1. redistribucionDeLosFrutos:}
	Calcula los recursos que obtiene cada uno de los individuos luego de que se redistribuyen
	los recursos del fondo monetario comun en partes iguales. El fondo monetario comun se compone de la suma de recursos
	iniciales aportados por todas las personas que cooperan. La salida es la lista de recursos que tendra cada jugador.

	\begin{proc}{redistribucionDeLosFrutos}{\In recursos : \TLista{\float}, \In cooperan : \TLista{\bool}}{\TLista{\float}}
		\hfill 

		\aux{Prom}{recursos: \TLista{\float}, cooperan: \TLista{\float}}{\float}
		{ 
			\frac{\sum\limits_{i=0}^{ \longitud{ recursos }-1 } \IfThenElse{cooperan[i]}{recursos[i]}{0}}{\longitud{ recursos }}
		}

		\hfill 

		\pred{TodosPositivos}{l : \TLista{\float}}
		{
			\paraTodo{i}{\ent}{( 0\leq i < \longitud{l} ) \implicaLuego (l[i] \geq 0) }
		}

		\hfill 
		
		\requiere{ \longitud{ recursos } \geq  1					}
		\requiere{ \longitud{ recursos } = \longitud{ cooperan } 	}
		\requiere{ TodosPositivos( recursos )						}
		\requiere{ Promedio= Prom( recursos, cooperan )						}

		\hfill 

		\asegura{ \longitud{ res } = \longitud{ recursos } = \longitud{ cooperan } }

		% La comente porque me parecia grotezca
		% \asegura{ \paraTodo{i}{\ent}{( 0\leq i < \longitud{res} ) \implicaLuego ( (( Cooperan[i] = \True )\land( res[i] = Promedio(recursos, cooperan) )) \lor (( Cooperan[i] = \False )\land( res[i] = recursos[i] + Promedio(recursos, cooperan) )) ) } }
		
		\asegura{  
			\paraTodo{i}{\ent}
			{( 0\leq i < \longitud{res} ) \implicaLuego ( \IfThenElse{(cooperan[i] = \True)}{(res[i] = Promedio)}{(res[i] = recursos[i] + Promedio)} ) } 
		}

	\end{proc}
	
	\subsection*{2. trayectoriaDeLosFrutosIndividualesALargoPlazo:}
	Actualiza (In/Out) la lista de 
	\textit{trayectorias}
	de los recursos de cada uno de los individuos. Inicialmente, cada una de las trayectorias (listas de recursos) contiene un único elemento que representa los recursos iniciales
	del individuo. El procedimiento agrega a las 
	\textit{trayectorias}
	los recursos que los individuos van obteniendo a medida que se van produciendo los resultados de los 
	\textit{eventos}
	en función de la lista de 
	\textit{pagos}
	que le ofrece la naturaleza (o casa de apuestas) a cada uno de los individuos, las 
	\textit{apuestas}
	(o inversiones) que realizan los individuos en cada paso temporal, y la lista de individuos que 
	\textit{cooperan}
	aportando al fondo monetario común.

	\begin{proc}{trayectoriaDeLosFrutosIndividualesALargoPlazo}{\Inout trayectorias : {\TLista{\TLista{\float}}} , \In cooperan : \TLista{\bool}, \In apuestas : {\TLista{\TLista{\float}}}, \In pagos : {\TLista{\TLista{\float}}}, \In eventos : \TLista{\TLista{ \mathbb{ N }}}}
		
		\hfill 

		\aux{Prom}{trayectorias : {\TLista{\TLista{\float}}}, fila : \mathbb{N} , apuestas : {\TLista{\TLista{\float}}} , pagos : {\TLista{\TLista{\float}}}, eventos : {\TLista{\TLista{\mathbb{N}}}}, cooperan : {\TLista{bool}}}
		
		
		{ 
			\frac{\sum\limits_{i=0}^{ \longitud{cooperan}-1 } \IfThenElse{cooperan[i]}{(apuestas[i][eventos[i][fila]]*pagos[eventos[i][fila]]*trayectorias[i][i][fila])}{0}}
			
			{\longitud{cooperan}}
		}

		\hfill

		
		\requiere { trayectorias = T_0}
		\requiere { \longitud{ T_0 } = \longitud{ eventos } = \longitud{ cooperan} = \longitud{ pagos } = \longitud{ apuestas} }
		\requiere { esMatriz(T_0) \land esMatriz(eventos) \land esMatriz(apuestas) \land esMatriz(pagos)}
		\requiere { \longitud{ T_0} \geq 1  \wedge  filas(trayectorias) = 1 } 
		\requiere { filas(apuestas) = filas(pagos) }
		\requiere { (\forall  pago: \float)(pago \in pagos \longrightarrow 0 \leq pago)} 
		\requiere { (\forall  trayectoria: \float)(trayectoria \in T_0 \longrightarrow 0 \leq trayectoria)}
		\requiere { (\forall  evento: \ent)(evento \in eventos \longrightarrow 0 \leq evento < \longitud{filas(eventos)})}
		\requiere { (\forall  apuesta: \float)(apuesta \in 0 \leq apuesta \leq 1)}
		\requiere { (\forall  i: \ent)(0 \leq i < \longitud{apuestas} \implicaLuego \sum\limits_{i=0}^{ \longitud{ apuestas }-1 } apuestas[i] = 1 )}                    


		\hfill
		

		\asegura {\longitud{trayectorias} = \longitud{T_0} \yLuego (\forall individuo:\ent)(0 \leq individuo < \longitud{T_0} \implicaLuego trayectorias[individuo][0] = T_0[individuo][0])}
		\asegura {filas(trayectorias) - 1 = filas(eventos)}
		\asegura { (\forall individuo:\ent)(0 \leq individuo < \longitud{T_0}) \yLuego (\forall paso:\ent)(1 \leq paso \leq filas(eventos)) \implicaLuego
		\yLuego 
		
		\textit{Promedio} = Prom (trayectorias, paso - 1, apuestas, pagos, eventos, cooperan)
		\implicaLuego 
		
		(trayectorias[individuo][paso] = \textit{Promedio}) \longleftrightarrow (cooperan[individuo] = \textit{true})
		
		\oLuego 
		
		(trayectorias[individuo][paso] =
		
		apuestas[individuo][eventos[individuo[paso-1]]] * pagos[individuo][eventos[individuo[paso-1]]] * 
		
		trayectorias[individuo][paso-1] + \textit{Promedio}) \longleftrightarrow (cooperan[individuo] = \textit{false})}

	\end{proc}

	\subsection*{3. trayectoriaExtrañaEscalera}
	Esta función devuelve True sii en la trayectoria de un individuo existe un único punto mayor a sus vecinos (llamado máximo local). Un elemento es máximo local si es mayor estricto que sus vecinos inmediatos.



	\begin{proc}{trayectoriaExtrañaEscalera}{\In trayectoria : \TLista{\float}}{\bool}
		
		\hfill

		\requiere {\longitud{trayectoria} > 1 \wedge TodosPositivos(trayectoria)}


		\asegura {res = \textit{true} \leftrightarrow esÓExcluyente(esÓExcluyenteExcluyente((trayectoria[0]> trayectoria), trayectoria[\longitud{trayectoria}-1]> trayectoria[\longitud{trayectoria}-2]), 
		(\exists i: \ent)(\sum_{i=1}^{\longitud{trayectoria}-2} \IfThenElse{trayectoria[i] > trayectoria [i+1] \yLuego trayectoria[i] > trayectoria[i-1]}{1}{0}=1))}


		\pred{esÓExcluyente}{x : \bool , y : \bool}
		{(x \wedge \neg y) \lor (\neg x \wedge y)}

		
	\end{proc}

\end{document}
