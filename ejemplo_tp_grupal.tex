\documentclass[10pt,a4paper]{article}

\input{AEDmacros}
\usepackage{caratula} % Version modificada para usar las macros de algo1 de ~> https://github.com/bcardiff/dc-tex


\titulo{Trabajo practico 1: Especificacion y WP}
\subtitulo{Primer cuatrimestre de 2024}

\fecha{\today}

\materia{Algoritmos y Estructuras de Datos - DC - UBA}
\grupo{Grupo: EVLUAFGUEYVCXPKHDNUP}

\integrante{Curti, Nahuel}{97/23}{nahuel0curti@gmail.com}
\integrante{Dosio, Martin}{291/23}{dosiomartin@gmail.com}
\integrante{Lemes, Tiziano}{796/23}{tizilemes@gmail.com}
\integrante{Rizzi, Francisco}{766/23}{rizzifranciscojose@gmail.com}
% Pongan cuantos integrantes quieran

% Declaramos donde van a estar las figuras
% No es obligatorio, pero suele ser comodo
\graphicspath{{../static/}}

\begin{document}

\maketitle

\section{1. Especificacion}




\subsection*{1. redistribucionDeLosFrutos:}
	Calcula los recursos que obtiene cada uno de los individuos luego de que se redistribuyen
	los recursos del fondo monetario comun en partes iguales. El fondo monetario comun se compone de la suma de recursos
	iniciales aportados por todas las personas que cooperan. La salida es la lista de recursos que tendra cada jugador.

	\begin{proc}{redistribucionDeLosFrutos}{\In recursos : \TLista{\float}, \In cooperan : \TLista{\bool}}{\TLista{\float}}
		\hfill 

		\aux{Prom}{Rec: \TLista{\float}, Coop: \TLista{\float}}{\float}
		{ 
			\frac{\sum\limits_{i=0}^{ \longitud{ Rec }-1 } \IfThenElse{Coop[i]}{Rec[i]}{0}}{\longitud{ Rec }}
		}

		\hfill 

		\pred{TodosPositivos}{l : \TLista{\float}}
		{
			\paraTodo{i}{\ent}{( 0\leq i < \longitud{l} ) \implicaLuego (l[i] \geq 0) }
		}

		\hfill 
		
		\requiere{ \longitud{ recursos } \geq  1					}
		\requiere{ \longitud{ recursos } = \longitud{ cooperan } 	}
		\requiere{ TodosPositivos( recursos )						}
		\requiere{ Promedio= Prom( recursos, cooperan )						}

		\hfill 

		\asegura{ \longitud{ res } = \longitud{ recursos } = \longitud{ cooperan } }

		% La comente porque me parecia grotezca
		% \asegura{ \paraTodo{i}{\ent}{( 0\leq i < \longitud{res} ) \implicaLuego ( (( Cooperan[i] = \True )\land( res[i] = Promedio(recursos, cooperan) )) \lor (( Cooperan[i] = \False )\land( res[i] = recursos[i] + Promedio(recursos, cooperan) )) ) } }
		
		\asegura{  
			\paraTodo{i}{\ent}
			{( 0\leq i < \longitud{res} ) \implicaLuego ( \IfThenElse{(cooperan[i] = \True)}{(res[i] = Promedio)}{(res[i] = recursos[i] + Promedio)} ) } 
		}

	\end{proc}


\end{document}
